%%%%%%%%%%%%%%%%%%%%%%%%%%%%%%%%%%%%%%%%%
% Arsclassica Article
% LaTeX Template
% Version 1.1 (1/8/17)
%
% This template has been downloaded from:
% http://www.LaTeXTemplates.com
%
% Original author:
% Lorenzo Pantieri (http://www.lorenzopantieri.net) with extensive modifications by:
% Vel (vel@latextemplates.com)
%
% License:
% CC BY-NC-SA 3.0 (http://creativecommons.org/licenses/by-nc-sa/3.0/)
%
%%%%%%%%%%%%%%%%%%%%%%%%%%%%%%%%%%%%%%%%%

%----------------------------------------------------------------------------------------
%	PACKAGES AND OTHER DOCUMENT CONFIGURATIONS
%----------------------------------------------------------------------------------------





\documentclass[
10pt, % Main document font size
a4paper, % Paper type, use 'letterpaper' for US Letter paper
oneside, % One page layout (no page indentation)
%twoside, % Two page layout (page indentation for binding and different headers)
headinclude,footinclude, % Extra spacing for the header and footer
BCOR5mm, % Binding correction
]{scrartcl}


\usepackage{listings}
\usepackage{color}

\definecolor{dkgreen}{rgb}{0,0.6,0}
\definecolor{gray}{rgb}{0.5,0.5,0.5}
\definecolor{mauve}{rgb}{0.58,0,0.82}

\lstset{frame=tb,
	language=C,
	aboveskip=3mm,
	belowskip=3mm,
	showstringspaces=false,
	columns=flexible,
	basicstyle={\small\ttfamily},
	numbers=none,
	numberstyle=\tiny\color{gray},
	keywordstyle=\color{blue},
	commentstyle=\color{dkgreen},
	stringstyle=\color{mauve},
	breaklines=true,
	breakatwhitespace=true,
	tabsize=3
}



\usepackage{german}

%usepackage[utf8]{inputenc}
%\usepackage{geometry}
\usepackage[german,onelanguage,linesnumbered, ruled]{algorithm2e}
\SetAlFnt{\small}
\SetAlCapFnt{\large}
\SetAlCapNameFnt{\large}
%\usepackage{algpseudocode}


\input{structure.tex} % Include the structure.tex file which specified the document structure and layout

\hyphenation{Fortran hy-phen-ation} % Specify custom hyphenation points in words with dashes where you would like hyphenation to occur, or alternatively, don't put any dashes in a word to stop hyphenation altogether

%----------------------------------------------------------------------------------------
%	TITLE AND AUTHOR(S)
%----------------------------------------------------------------------------------------

\title{\normalfont\spacedallcaps{Projektaufgabe pixCheckTangle}} % The article title

%\subtitle{Subtitle} % Uncomment to display a subtitle

\author{\spacedlowsmallcaps{Raphael Drechsler}} % The article author(s) - author affiliations need to be specified in the AUTHOR AFFILIATIONS block

\date{} % An optional date to appear under the author(s)

%----------------------------------------------------------------------------------------

\begin{document}

%----------------------------------------------------------------------------------------
%	HEADERS
%----------------------------------------------------------------------------------------

\renewcommand{\sectionmark}[1]{\markright{\spacedlowsmallcaps{#1}}} % The header for all pages (oneside) or for even pages (twoside)
%\renewcommand{\subsectionmark}[1]{\markright{\thesubsection~#1}} % Uncomment when using the twoside option - this modifies the header on odd pages
\lehead{\mbox{\llap{\small\thepage\kern1em\color{halfgray} \vline}\color{halfgray}\hspace{0.5em}\rightmark\hfil}} % The header style

\pagestyle{scrheadings} % Enable the headers specified in this block

%----------------------------------------------------------------------------------------
%	TABLE OF CONTENTS & LISTS OF FIGURES AND TABLES
%----------------------------------------------------------------------------------------

\maketitle % Print the title/author/date block

\setcounter{tocdepth}{2} % Set the depth of the table of contents to show sections and subsections only

\tableofcontents % Print the table of contents

\listoffigures % Print the list of figures

\listoftables % Print the list of tables




%----------------------------------------------------------------------------------------

\newpage % Start the article content on the second page, remove this if you have a longer abstract that goes onto the second page

%----------------------------------------------------------------------------------------
%	INTRODUCTION
%----------------------------------------------------------------------------------------
\section{Problembeschreibung}

Gegeben ist ein quadratisches Raster von n x n Feldern. Jedes der Felder kann schwarz oder weiß gefärbt sein. 
Es ist ein Algorithmus zu implementieren, welcher

\begin{itemize}[noitemsep] % [noitemsep] removes whitespace between the items for a compact look
	\item eine Einfache Eingabe eines solche Rasters erlaubt
	\item als Rechteck zusammenhängende Felder im Raster erkennt 
	\item die resultierenden Rechtecke ausgibt
\end{itemize}

Dabei soll der Algorithmus für die Ausführung auf dem Cluster-System implementiert werden.\\
Anschließend soll mittels Laufzeitmessungen die Effizienz der Parallelisierung betrachtet werden. Dafür ist es erforderlich den Algorithmus als sequentiellen Algorithmus ausführen zu können.

\section{Beschreibung der Implementierung}

Die Implementierung ist in zwei Programmen umgesetzt. 

\begin{itemize}[noitemsep] % [noitemsep] removes whitespace between the items for a compact look
	\item \textit{gridGenerator.c}: Programm zum generieren von \textit{.txt}-Dateien, in denen das Raster gespeichert ist.  
	\item \textit{pixCheckTangle.c}: Programm zur Überprüfung des als \textit{.txt}-Datei übergebenen Rasters auf Rechtecke.
\end{itemize}

In den Folgenden Abschnitten wird die Funktionalität der Programme beschrieben.

\subsection{gridGenerator: Generieren der Files}

Das Programm \textit{gridGenerator.c} wird per mpirun-Befehl über die Konsole aufgerufen.\\

\begin{lstlisting}
mpirun gridGenerator.c [gridfile.txt]
\end{lstlisting}

Wird dabei eine zuvor durch den gridGenerator erzeugte .txt-Datei als Parameter angegeben, kann diese Datei bearbeitet werden. Andernfalls wird eine neue Datei erstellt. Der Programm-Ablauf ist im Folgenden als Nassi-Shneiderman-Diagramm dargestellt.\\
\\Der Vollstädige Code ist im Anhang ab Seite (DRT oder Verweis angeben)... gelistet. 
\pagebreak

\begin{figure}[h]
	\centering 
	\includegraphics[width=0.9\columnwidth]{Diag_1} 
	\caption[Funktionalität von \textit{gridGenerator.c}]{Funktionalität von \textit{gridGenerator.c} dargestellt als Nassi-Shneiderman-Diagramm}
\end{figure}


\subsection{pixCheckTangle: Beschreibung des Sequentiellen Algorithmus}

Das Programm \textit{pixCheckTangle.c} wird per mpirun-Befehl über die Konsole aufgerufen. Dabei ist als dem Aufruf eine Raster-Textdatei als Argument zu übergeben. Die übergebene Datei wird dann auf Rechtecke überprüft.\\

\begin{lstlisting}
mpirun pixCheckTangle.c gridfile.txt
\end{lstlisting}

Für den Fall, dass nur ein Prozessor an der Ausführung des Programmes beteiligt ist, wird die sequentielle Variante des Algorithmus ausgeführt. Diese ist im Folgenden als BPMN-Daigramm dargestellt.

\begin{figure}[h]
	\centering 
	\includegraphics[width=0.8\columnwidth]{Diag_2f} 
	\caption[Funktionalität von \textit{pixCheckTangle.c} sequentiell]{Sequentieller Ablauf von \textit{pixCheckTangle.c} dargestellt als BPMN-Diagramm }
	
\end{figure}

Der vollständige Code zum Algorithmus ist im Anhang enthalten. Die Hauptfunktionalität findet sich dabei in dem Codeabschnitt, welcher ausgeführt wird,   wenn der ausführende Prozess den Rang 0 hat und es nur einen Prozess gibt.
%
%
%\begin{figure}[h]
%	\centering 
%
%	\begin{lstlisting}
%	if(rank == 0){
%			if (noOfProcs==1){
%					//run the sequential porcess
%			} ...
%	\end{lstlisting}
%	\caption[Einordnung \textit{pixCheckTangle.c} sequentiell im Quellcode]{Einordnung des sequentiellen Ablaufs von \textit{pixCheckTangle.c} im Quellcode}
%	
%\end{figure}


Die Funktionalität zum Ermitteln der Rechtecke, wird im Kapitel 2.4 näher beschrieben. Wie die Funktionalitäten zur Zeitmessung beschaffen sind, wird in Kapitel 3.1 beschrieben.


\subsection{pixCheckTangle: Beschreibung des Cluster-Algorithmus}

Für den Fall, dass mehrere Prozessoren an der Abarbeitung des Programmes beteiligt sind, wird die Cluster-Variante des Algorithmus ausgeführt.\\
Dabei übernimmt der Prozess mir dem Rang 0 die Rolle des Master-Prozesses. Alle übrigen Prozesse übernehmen die Rolle von Worker-Prozessen. \\
Das Raster wird dann in vom Master-Prozess in mehrere Teile zerlegt. Jeder Worker-Prozess übernimmt dann die Abarbeitung eines Teil-Rasters. \\

\textbf{Beispiel für Verarbeitung durch Cluster-Prozess}

Beispielsweise wird das folgende Raster betrachtet. Das Zeichen \textit{\#} repräsentiert dabei ein schwarzes Feld, weiße Felder werden durch Leerzeichen dargestellt.
\begin{figure}[h]
	\centering 
	\includegraphics[width=0.25\columnwidth]{Raster_1} 
	\caption[Cluster-Prozess: Beispiel-Raster]{Ein Raster-Beispiel }
\end{figure}

Für den Fall, das an der Abarbeitung des Programmes 4 Prozesse beteiligt sind, wird das Raster vom Master-Prozess in drei Teil-Raster geteilt. Die jeweiligen Teile werden dann den Worker-Prozessen zugewiesen und von diesen abgearbeitet.

\begin{figure}[h]
	\centering 
	\includegraphics[width=0.6\columnwidth]{Raster_2} 
	\caption[Cluster-Prozess: Aufteilung Beispiel-Raster mit 4 Prozessen]{Aufteilung des Beispiel-Rasters bei 4 Prozessen}
\end{figure}

Die Worker-Prozesse können nun feststellen, ob es sich bei einzelnen/zusammenhängenden Pixeln (im Folgenden als Figur bezeichnet) um Rechtecke handelt oder nicht. In der Folgenden Grafik, werden Figuren, die kein Rechteck darstellen mit einem \textit{X} dargestellt. Rechtecke werden weiterhin durch das Zeichen \textit{\#} repräsentiert.

\begin{figure}[h]
	\centering 
	\includegraphics[width=0.9\columnwidth]{Raster_3} 
	\caption[Cluster-Prozess: Worker-Prozess: Nicht-Rechteck-Figuren]{Worker-Prozesse identifizieren Nicht-Rechteck-Figuren}
\end{figure}

Wenn ein Rechteck in einem Teil-Raster zu einem Rand der Teil-Rasters reicht und an diesem Rand im gesamten Raster ein weiteres Raster angrenzt, so könnte die Figur im angrenzenden Teil-Raster fortgesetzt werden. In diesem Fall ist also durch das Verarbeiten des Teil-Rasters keine Aussage darüber möglich, ob es sich um ein tatsächlich um ein Rechteck handelt. Entsprechende Rechtecke (Im Weiteren als potentielle Rechtecke) werden in der folgenden Grafik als \textit{?} gekennzeichnet.\\
Das Ergebnis der Abarbeitung der Teil-Raster in den Worker-Prozessen sieht also wie folgt aus:

\begin{figure}[h]
	\centering 
	\includegraphics[width=0.9\columnwidth]{Raster_4} 
	\caption[Cluster-Prozess: Worker-Prozess: Rechtecke und potentielle Rechtecke]{Worker-Prozesse identifizieren Rechtecke und potentielle Rechtecke}
\end{figure}

Nach dem Verarbeiten der Teil-Raster durch die Worker-Prozesse, muss die finale Überprüfung der potentiellen Rechtecke vom Master-Prozess übernommen werden.

\begin{figure}[h]
	\centering 
	\includegraphics[width=0.7\columnwidth]{Raster_5} 
	\caption[Cluster-Prozess: Beispiel-Raster]{Ein Raster-Beispiel }
\end{figure}

Damit liegt dem Master-Prozess nun das vollständig überprüfte Raster vor. Die Einzelnen Rechtecke können nun dem Benutzer ausgegeben werden.\\

\textbf{Kommunikation und Ablauf im Cluster-Prozess }\\
Wie dabei Arbeits-Aufteilung und die Kommunikation zwischen den Prozessen verläuft, sei im Folgenden als BPMN-Diagramm dargestellt.\\

Der vollständige Code findet sich im Anhang. Die wesentlichen Funktionalitäten für den Master-Prozess finden sich dabei in dem Codeabschnitt, welcher ausgeführt wird, wenn der ausführende Prozess den Rang 0 hat und es mehr als einen Prozess gibt. Der Code für die Worker-Prozesse findet sich im Abschnitt, der ausgeführt wird, wenn der Rang des Prozesses nicht 0 ist.\\

Die Funktionalität zum Ermitteln der Rechtecke, wird im Kapitel 2.4 näher beschrieben. Wie die Funktionalitäten zur Zeitmessung beschaffen sind, wird in Kapitel 3.1 beschrieben.

\pagebreak

\begin{figure}[h]
	\centering 
	\includegraphics[width=1\columnwidth]{Diag_3f} 
	\caption[Funktionalität von \textit{pixCheckTangle.c} Cluster-Prozess]{Ablauf des Cluster-Prozesses von \textit{pixCheckTangle.c} dargestellt als BPMN-Diagramm }
	
\end{figure}



\subsection{pixCheckTangle: Funktion für das Abarbeiten eines (Teil-)Rasters}

Im Wesentlichen folgt das Vorgehen für die Unterscheidung zwischen Rechtecken und Nicht-Rechteck-Figuren dem folgenden Prinzip:

\begin{enumerate}[noitemsep]
	\item Erkenne Spalten-Reichweite der betrachteten Figur anhand der erster Zeile
	\item Ist In Zeile über der Figur innerhalb der Spalten-Reichweite ein Feld schwarz, ist die Figur kein Rechteck
	\item Enthält eine Zeile unterhalb der ersten innerhalb der Spalten-Reichweite der Figur schwarze und weiße Felder, ist die Figur kein Rechteck
	\item Grenzt an eine schwarze Zeile der Figur (von Spalten-Reichweite eingegrenzt) ein schwarzes Feld, ist die Figur kein Rechteck
\end{enumerate}
Zusätzlich ist dabei zu berücksichtigen, ob es sich bei einem erkannten Rechteck, wie weiter oben beschrieben, nur um ein potentielles Rechteck ist.\\
Die Implementierung der Funktion folgt danach dem folgenden, als Pseudo-Code dargestellten Algorithmus:\\

\begin{algorithm}[H]

	\KwData{$Raster$, $InfoUeberAngenzedeTeilraster$}
	\KwResult{RasterVerarbeitet }
	\ForEach{$Feld$ in $Raster$ zeilenweise gelesen}{
		\If{$Feld$ ist Schwarz und nicht markiert}{
			$FigurIstRechteck \leftarrow wahr;$\\ 
			$startZeile \leftarrow aktuelle Zeile;$\\
			$startSpalte \leftarrow aktuelle Spalte;$\\
			$endeZeile \leftarrow aktuelle Zeile;$\\
			$endeSpalte \leftarrow$ Spalte von letzem $Feld$ in $startZeile$, das mit aktuellem $Feld$ zusammenhängt$;$\\
			\uIf{In Zeile über $startZeile$ im Bereich von $startSpalte$ bis $endeSpalte$ ist ein schwarzes Feld}{
				$FigurIstRechteck \leftarrow falsch;$\\
			}
			\Else{
				$RechteckGeprueft \leftarrow falsch;$\\
				\While{(nicht $FigurIstRechteck )$ und (nicht $RechteckGeprueft$)}{
					
					
					\uIf{Zeile unter $endeZeile$ enthält schwarze und weiße Felder}{
						$FigurIstRechteck \leftarrow falsch;$\\
				     }
				    \Else {
				    	\uIf{Zeile unter $endeZeile$ enthält nur schwarze Felder}{
				    		\uIf{In Zeile unter $endeZeile$ ist feld links von $startSpalte$ oder/und rechts $endeSpalte$ von schwarz}{
				    			$FigurIstRechteck \leftarrow falsch;$\\
				    		}
				    		\Else{
				    			$endeZeile \leftarrow endeZeile +1;$\\
				    		}
				    	}
				    	\Else{
				    		$RechteckGeprueft \leftarrow wahr;$\\
				    	}
				    }
				}
			}
		\uIf{$FigurIstRechteck$}{
			\uIf{$Figur$ liegt an Raster-Rand, der an weiterem Teil-Raster angrenzt}{
				Markiere alle schwarzen $Felder$ innerhalb der Reichweite, die von $startZeile$,$endeZeile$,$startSpalte$ und $endeSpalte$ aufgespannt wird als potentielles Rechteck-Feld;
			}
			\Else{
				Markiere alle schwarzen $Felder$ innerhalb der Reichweite, die von $startZeile$,$endeZeile$,$startSpalte$ und $endeSpalte$ aufgespannt wird als Rechteck-Feld;
			}
		}
		\Else{
			Markiere alle schwarzen $Felder$ innerhalb der Reichweite, die von $startZeile$,$endeZeile$,$startSpalte$ und $endeSpalte$ aufgespannt wird als Nicht-Rechteck-Feld;
		}
		}
	}	
\end{algorithm}\
\\
Der Code für die Funktionalität findet sich im Anhang im Programm \textit{pixCheckTangle.c} in der Funktion \textit{processSubgrid}.

\subsection{pixCheckTangle: Funktion für das Festlegen potentieller Rechtecke}

Überlegung: Zu prüfen: die zwei zeilen am bruch
dann von unten von oben 


- Darstellung als Algorithmus im folhenden, dabei Info ob oben o untn in par IstObererZeile

\begin{algorithm}[H]
	
	\KwData{$Raster$,$zuPruefendeZeile$ $IstObererZeile$}
	\KwResult{RasterVerarbeitet }
	\For{Jedes schwarze Feld in $zuPruefendeZeile$ mit Markierung = potentielles-Rechteck-Feld}{
		$startZeile \leftarrow aktuelle Zeile;$\\
		$startSpalte \leftarrow aktuelle Spalte;$\\
		$endeZeile \leftarrow aktuelle Zeile;$\\
		$endeSpalte \leftarrow$ Spalte von letzem $Feld$ in $startZeile$, das mit aktuellem $Feld$ zusammenhängt$;$\\
		\uIf{$IstObererZeile$}{}
	}
\end{algorithm}\

Der Code für die Funktionalität findet sich im Anhang im Programm \textit{pixCheckTangle.c} in der Funktion \textit{checkPotentialRectangle}.


\section{Messungen}
\subsection{Vorgehen beim Messen}

Ohne NWK. Was heißt das?
Mit:

\begin{figure}[h]
	\centering 
	\includegraphics[width=0.5\columnwidth]{Time_1} 
	\caption[DRT]{DRT }
	
\end{figure}

Ohne sieht so aus:

\begin{figure}[h]
	\centering 
	\includegraphics[width=0.5\columnwidth]{Time_2} 
	\caption[DRT]{DRT }
	
\end{figure}

Also: Bisschen Messen mit 

\begin{lstlisting}
MPI_Wtimetick?
\end{lstlisting}

Worker-Prozesse auch messen und nach Zeitmessung und Algo zusammenführen:

\begin{lstlisting}
MPI_Reduce
\end{lstlisting}


\subsection{Messergebnisse}

\subsection{Interpretation der Messergebnisse}

\section{Fazit}
\subsection{Einschätzung des Nutzens einer Parallelisierung}
\subsection{Mögliche Optimierungen}

\section{bla}

A statement requiring citation \cite{Figueredo:2009dg}.

\lipsum[1-3] % Dummy text

Some mathematics in the text: $\cos\pi=-1$ and $\alpha$.
 
%----------------------------------------------------------------------------------------
%	METHODS
%----------------------------------------------------------------------------------------

\section{Methods}

\lipsum[5] % Dummy text

\begin{enumerate}[noitemsep] % [noitemsep] removes whitespace between the items for a compact look
\item First item in a list
\item Second item in a list
\item Third item in a list
\end{enumerate}

%------------------------------------------------

\subsection{Paragraphs}

\lipsum[6] % Dummy text

\paragraph{Paragraph Description} \lipsum[7] % Dummy text

\paragraph{Different Paragraph Description} \lipsum[8] % Dummy text

%------------------------------------------------

\subsection{Math}

\lipsum[4] % Dummy text

\begin{equation}
\cos^3 \theta =\frac{1}{4}\cos\theta+\frac{3}{4}\cos 3\theta
\label{eq:refname2}
\end{equation}

\lipsum[5] % Dummy text

\begin{definition}[Gauss] 
To a mathematician it is obvious that
$\int_{-\infty}^{+\infty}
e^{-x^2}\,dx=\sqrt{\pi}$. 
\end{definition} 

\begin{theorem}[Pythagoras]
The square of the hypotenuse (the side opposite the right angle) is equal to the sum of the squares of the other two sides.
\end{theorem}

\begin{proof} 
We have that $\log(1)^2 = 2\log(1)$.
But we also have that $\log(-1)^2=\log(1)=0$.
Then $2\log(-1)=0$, from which the proof.
\end{proof}

%----------------------------------------------------------------------------------------
%	RESULTS AND DISCUSSION
%----------------------------------------------------------------------------------------

\section{Results and Discussion}

Reference to Figure~\vref{fig:gallery}. % The \vref command specifies the location of the reference

\begin{figure}[tb]
\centering 
\includegraphics[width=0.5\columnwidth]{GalleriaStampe} 
\caption[An example of a floating figure]{An example of a floating figure (a reproduction from the \emph{Gallery of prints}, M.~Escher,\index{Escher, M.~C.} from \url{http://www.mcescher.com/}).} % The text in the square bracket is the caption for the list of figures while the text in the curly brackets is the figure caption
\label{fig:gallery} 
\end{figure}

\lipsum[10] % Dummy text

%------------------------------------------------

\subsection{Subsection}

\lipsum[11] % Dummy text

\subsubsection{Subsubsection}

\lipsum[12] % Dummy text

\begin{description}
\item[Word] Definition
\item[Concept] Explanation
\item[Idea] Text
\end{description}

\lipsum[12] % Dummy text

\begin{itemize}[noitemsep] % [noitemsep] removes whitespace between the items for a compact look
\item First item in a list
\item Second item in a list
\item Third item in a list
\end{itemize}

\subsubsection{Table}

\lipsum[13] % Dummy text

\begin{table}[hbt]
\caption{Table of Grades}
\centering
\begin{tabular}{llr}
\toprule
\multicolumn{2}{c}{Name} \\
\cmidrule(r){1-2}
First name & Last Name & Grade \\
\midrule
John & Doe & $7.5$ \\
Richard & Miles & $2$ \\
\bottomrule
\end{tabular}
\label{tab:label}
\end{table}

Reference to Table~\vref{tab:label}. % The \vref command specifies the location of the reference

%------------------------------------------------

\subsection{Figure Composed of Subfigures}

Reference the figure composed of multiple subfigures as Figure~\vref{fig:esempio}. Reference one of the subfigures as Figure~\vref{fig:ipsum}. % The \vref command specifies the location of the reference

\lipsum[15-18] % Dummy text

\begin{figure}[tb]
\centering
\subfloat[A city market.]{\includegraphics[width=.45\columnwidth]{Lorem}} \quad
\subfloat[Forest landscape.]{\includegraphics[width=.45\columnwidth]{Ipsum}\label{fig:ipsum}} \\
\subfloat[Mountain landscape.]{\includegraphics[width=.45\columnwidth]{Dolor}} \quad
\subfloat[A tile decoration.]{\includegraphics[width=.45\columnwidth]{Sit}}
\caption[A number of pictures.]{A number of pictures with no common theme.} % The text in the square bracket is the caption for the list of figures while the text in the curly brackets is the figure caption
\label{fig:esempio}
\end{figure}

%----------------------------------------------------------------------------------------
%	BIBLIOGRAPHY
%----------------------------------------------------------------------------------------

\renewcommand{\refname}{\spacedlowsmallcaps{References}} % For modifying the bibliography heading

\bibliographystyle{unsrt}

\bibliography{sample.bib} % The file containing the bibliography

%----------------------------------------------------------------------------------------

\end{document}